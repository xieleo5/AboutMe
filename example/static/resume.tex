\documentclass[letterpaper,11pt]{article}

\usepackage{latexsym}
\usepackage[empty]{fullpage}
\usepackage{titlesec}
\usepackage{marvosym}
\usepackage[usenames,dvipsnames]{color}
\usepackage{verbatim}
\usepackage{enumitem}
\usepackage[colorlinks=true, linkcolor=blue]{hyperref}
\usepackage{fancyhdr}
\usepackage[english]{babel}
\usepackage{tabularx}
\usepackage{fontawesome5}
\usepackage{multicol}
\setlength{\multicolsep}{-3.0pt}
\setlength{\columnsep}{-1pt}
\input{glyphtounicode}


%----------FONT OPTIONS----------
% sans-serif
% \usepackage[sfdefault]{FiraSans}
% \usepackage[sfdefault]{roboto}
% \usepackage[sfdefault]{noto-sans}
% \usepackage[default]{sourcesanspro}

% serif
\usepackage{CormorantGaramond}
\usepackage{charter}


\pagestyle{fancy}
\fancyhf{} % clear all header and footer fields
\fancyfoot{}
\renewcommand{\headrulewidth}{0pt}
\renewcommand{\footrulewidth}{0pt}

% Adjust margins
\addtolength{\oddsidemargin}{-0.6in}
\addtolength{\evensidemargin}{-0.5in}
\addtolength{\textwidth}{1.19in}
\addtolength{\topmargin}{-.7in}
\addtolength{\textheight}{1.4in}

\urlstyle{same}

\raggedbottom
\raggedright
\setlength{\tabcolsep}{0in}

% Sections formatting
\titleformat{\section}{
  \vspace{-8pt}\raggedright\Large\bfseries
}{}{0em}{}[\color{black}\titlerule \vspace{-5pt}]

% Ensure that generate pdf is machine readable/ATS parsable
\pdfgentounicode=1

%-------------------------
% Custom commands
\newcommand{\resumeItem}[1]{
  \item\small{
    {\raggedright #1 \vspace{-2pt}}
  }
}

\newcommand{\classesList}[4]{
    \item\small{
        {#1 #2 #3 #4 \vspace{-2pt}}
  }
}

\newcommand{\resumeSubheading}[5]{
  \vspace{-2pt}\item
    \begin{tabular*}{1.0\textwidth}[t]{l@{\extracolsep{\fill}}r}
      \textbf{#1} \textit{#2} & \textbf{\small #3} \\
      \textit{\small#4} & \textit{\small #5} \\
    \end{tabular*}\vspace{-7pt}
}

\newcommand{\resumePubHeading}[3]{
  \vspace{-2pt}\item
    \begin{tabular*}{1.0\textwidth}[t]{l@{\extracolsep{\fill}}r}
      \textbf{#1} & \textbf{\small{#3}} \\
    \end{tabular*}\\
    \vspace*{1pt}
    \small{#2}\\\vspace{-7pt}
}

\newcommand{\resumeSubSubheading}[2]{
    \item
    \begin{tabular*}{0.97\textwidth}{l@{\extracolsep{\fill}}r}
      \textit{\small#1} & \textit{\small #2} \\
    \end{tabular*}\vspace{-7pt}
}

\newcommand{\resumeProjectHeading}[2]{
    \item
    \begin{tabular*}{1.001\textwidth}{l@{\extracolsep{\fill}}r}
      \small#1 & \textbf{\small #2}\\
    \end{tabular*}\vspace{-7pt}
}

\newcommand{\resumeSubItem}[1]{\resumeItem{#1}\vspace{-4pt}}

\renewcommand\labelitemi{$\vcenter{\hbox{\tiny$\bullet$}}$}
\renewcommand\labelitemii{$\vcenter{\hbox{\tiny$\bullet$}}$}

\newcommand{\resumeSubHeadingListStart}{\begin{itemize}[leftmargin=0.0in, label={}]}
\newcommand{\resumeSubHeadingListEnd}{\end{itemize}}
\newcommand{\resumeItemListStart}{\begin{itemize}}
\newcommand{\resumeItemListEnd}{\end{itemize}\vspace{-3pt}}

%-------------------------------------------
%%%%%%  RESUME STARTS HERE  %%%%%%%%%%%%%%%%%%%%%%%%%%%%


\begin{document}

%----------HEADING----------
% \begin{tabular*}{\textwidth}{l@{\extracolsep{\fill}}r}
%   \textbf{\href{http://sourabhbajaj.com/}{\Large Sourabh Bajaj}} & Email : \href{mailto:sourabh@sourabhbajaj.com}{sourabh@sourabhbajaj.com}\\
%   \href{http://sourabhbajaj.com/}{http://www.sourabhbajaj.com} & Mobile : +1-123-456-7890 \\
% \end{tabular*}

\begin{center}
    {\Large \textbf{Yuqi Xie} } \\ \vspace{1pt}
    912 W 22nd Street Apt 208, Austin, TX, 78705 \\ \vspace{1pt}
    \small \raisebox{-0.1\height}\faPhone\ 737-341-4564 ~ \href{mailto:xieleo@utexas.edu}{\raisebox{-0.2\height}\faEnvelope\  \underline{xieleo@utexas.edu}} ~ 
    %\href{https://linkedin.com/in//}{\raisebox{-0.2\height}\faLinkedin\ \underline{linkedin.com/in/username}}  ~
    \href{https://xieleo5.github.io}{\raisebox{-0.2\height}\faGithub\ \underline{xieleo5.github.io}}
    \vspace{-8pt}
\end{center}


%-----------EDUCATION-----------
\section{Education}
\resumeSubHeadingListStart
    \resumeSubheading
      {University of Texas at Austin}{}{Sep. 2023 -- Present}
      {Master's in Computer Science - College of Natural Science}{Texas, U.S.}
      % \begin{itemize}
      %   \item \small Highlighted Courses: Robot Learning, Advanced topics in Computer Vision
      % \end{itemize}
    % \vspace*{5.0\multicolsep}
  \resumeSubHeadingListEnd
  \resumeSubHeadingListStart
    \resumeSubheading
      {University of Michigan, Ann Arbor}{}{Sep. 2021 -- Apr. 2023}
      {Bachelor of Computer Science - College of Engineering}{Michigan, U.S.}
    % \vspace*{5.0\multicolsep}
  \resumeSubHeadingListEnd
  \resumeSubHeadingListStart
    \resumeSubheading
      {Shanghai Jiao Tong University}{}{Sep. 2019 -- Aug. 2023}
      {Bachelor of Electrical and Computer Engineering - UM-SJTU Joint Institute}{Shanghai, China}
  \resumeSubHeadingListEnd

%-----------PROJECTS-----------
\section{Publication}
% \vspace{-15pt}
\resumeSubHeadingListStart
  \resumePubHeading
    {\textbf{Voyager: An Open-Ended Embodied Agent with Large Language Models}}
    {Guanzhi Wang, \textbf{Yuqi Xie}, Yunfan Jiang, Ajay Mandlekar, Chaowei Xiao, Yuke Zhu, Linxi Fan, Anima Anandkumar}
    {In submission, 2023. \href{https://arxiv.org/abs/2305.16291}{\underline{Arxiv}}}
    % a link to arxiv
    
    % \vspace{-15pt}
\resumeSubHeadingListEnd
\vspace{-5pt}
%-----------RESEARCH----------- 
\section{Research Experience}
  \resumeSubHeadingListStart
    % \resumeSubheading
    %   {UT Austin Robot Perception and Learning Lab}{}{August 2023 - Present}
    %   {Graduate Research Assistant $|$ \small Advised by Prof. Yuke Zhu}{Texas, U.S.}
    %   \resumeItemListStart
    %     \resumeItem{
    %       \textbf{Enhancing Realism of MetaHuman: A Fusion of Motion, Voice and Appearance}
          
    %       MetaHuman is released in Unreal Engine for building realistic human inside games. This project aims to build a MetaHuman that looks, talk and move like a real human to remove the uncanny valley of current 3D human models. \vspace*{-2pt}
    %       \begin{itemize}[leftmargin=*]
    %         \item Voice and facial expression: Use Azure Viseme API to generate facial position and sync through Unreal live link face. 
    %         \item General motion: Synthesis data using 3D mesh reconstruction from 2D video to train a motion control model.
    %         \item Appearance: The generation of realistic textures is achieved through utilizing Diffusion-based techniques.
    %       \end{itemize}
    %     }
    %   \resumeItemListEnd
    \resumeSubheading
    {Collaboration with NVIDIA and UT Austin}{}{April 2022 - August 2023}
    {Undergraduate Research Assistant $|$ \small Advised by Prof. Yuke Zhu, Dr. Linxi "Jim" Fan}{Remote}
    \resumeItemListStart
      \resumeItem{
        \textbf{Voyager: An Open-Ended Embodied Agent with Large Language Models}
        
        Voyager is the first lifelong learning agent that plays Minecraft purely in-context. It continuously improves itself by writing, refining, committing, and retrieving code from a skill library, all \textbf{without relying on gradient descent}.
        \vspace*{-2pt}
        \begin{itemize}[leftmargin=*]
            \item Developed the asynchronous backend server in JavaScript responsible for agent control and game observation.
            \item Leveraged OpenAI's GPT API to generate \textbf{Code as Policy} and used \textbf{Chain-of-Thought} prompting to improve code.
            \item Created the control primitives as both helper functions and examples of code writing.
            \item Designed the curriculum, action, self-verification, and skill library execution loop for the agent.
            \item Proposed the \textbf{warm-up schedule} and the \textbf{auto-resume mechanism} to optimize the training process.
            \item Conducted experiments on downstream unseen tasks and involved human interactions for building tasks.
            % \item Increased the observation data transition speed between the game and the agent by 2x using the gRPC protocol.
            % \item Enabled GPU acceleration for the rendering process of the \textit{Minecraft} game during training inside Docker containers.
        \end{itemize}
      }
      % \resumeItem{
      %   \textbf{Embodied AI agent based on \textsc{MineDojo2}}

      %   We aim to build a general-purpose embodied agent that can finish a wider range of zero-shot tasks under various circumstances in the game based on \textsc{MineDojo2} framework. My contributions include:
      %   \vspace*{-2pt}
      %   \begin{itemize}[leftmargin=*]
      %     \item Create baseline benchmark suites using the state-of-the-art OpenAI Video-PreTraining model.
      %     \item Apply Internet-scale pre-trained models on 300K Youtube videos in \textsc{MineDojo} to retrieve expert actions from human operations as data for large-scale training.
      %     \item Experiment with large-scale multi-modal Transformers on zero-shot human language driving \textit{Minecraft} tasks.
      %   \end{itemize}
      % }
      \resumeItem{
        \textbf{Auto Machine Learning Code Generation with Large Language Model} 
        
        Existing AutoML tools, such as Auto-Sklearn, are based on brute force searching and Bayesian optimization. Our goal is to train a code generation model to achieve an end-to-end approach from DataFrames to code for the best model.
        \vspace*{-2pt}
        \begin{itemize}[leftmargin=*]
          \item Collected and cleaned web-scaled data from solutions to Kaggle competitions.
          \item Synthesized data using the optimal model searched by current AutoML algorithm.
          \item Fine-tuned Codex model with \textbf{RLHF} using the accuracy of the generated code as the reward.
          % \item Increase the observation data transition speed between the game and the agent by 2x using on grpc protocol. 
          % \item Enable GPU acceleration for the rendering process of \textit{Minecraft} game during training inside Docker containers.
        \end{itemize}
      }
      \resumeItem{
        \textbf{Development for \textsc{MineDojo2}}
        
        \textsc{MineDojo} is a framework built on the popular \textit{Minecraft} game, featuring a simulation suite with thousands of diverse open-ended tasks and an Internet-scale knowledge base. 
        My contributions include: 
        \vspace*{-2pt}
        \begin{itemize}[leftmargin=*]
            \item Dramatically modified the backend infrastructure to allow the environment to run faster in a later version of the game, and making the environment more scalable for training a large-scale open-ended agent.
            \item Implemented a universal interface instead of cheating commands for an agent to complete zero-shot tasks like a human.
            % \item Increasing the observation data transition speed between the game and the agent by 2x using the gRPC protocol. 
            % \item Enabling GPU acceleration for the rendering process of the \textit{Minecraft} game during training inside Docker containers.
        \end{itemize}
    }
    \resumeItemListEnd  
    % \resumeSubheading
    % {SOCR MDP Team}{}{Winter and Fall Semester, 2022}
    %   {Undergraduate Research Assistant $|$ \small Advised by Prof. Ivo D. Dinov, Prof. Simeone Marino}{University of Michigan}
    %   \resumeItemListStart
    %     \resumeItem{\textbf{Datasifter---Sensitive Information Obfuscator}

    %     Datasifter is an obfuscator to prevent patients' sensitive information to be leaked from Hospital databases. The obfuscation is based on Machine Learning and Natural Language Processing technologies. My contributions includes:
    %     \vspace*{-2pt}
    %     \begin{itemize}[leftmargin=*]
    %       \item Experiment with pre-trained NLP models and ML metrics to implement the obfuscation part.
    %       \item Implement a script based on ML metrics to evaluate the effectiveness of data synthesis and data obfuscation.
    %       \item Improve the speed of the app by 10x by applying multi-thread programming.
    %     \end{itemize}
    %     }
    % \resumeItemListEnd
    
  \resumeSubHeadingListEnd
\vspace{-15pt}
%
%-----------PROJECTS-----------
\section{Projects}
  \resumeSubHeadingListStart
  \vspace{-5pt}
    \resumeProjectHeading
    {\textbf{Visual Language Model with Multi-modal Reasoning} $|$ \emph{Major Design Project}}{Summer Semester, 2023}
    \resumeItemListStart
      \resumeItem{Collected and synthesized multi-modal training data from open source datasets such as MIMIC-IT, VIST, VQA, .etc}
      \resumeItem{Fine-tuned \textbf{Vicuna-7B} with an image encoder and decoder from \textbf{Stable Diffusion} 2.1 on the datasets.}
      \resumeItem{Created a user interface for chatting with the model using image and text based on Gradio.}
    \resumeItemListEnd
    \vspace{-20pt}
    \resumeProjectHeading
    {\textbf{Insta485} $|$ \emph{Flask, React, JavaScript, Python, SQL, AWS}}{Winter Semester, 2021}
    \resumeItemListStart
      \resumeItem{Built the client side dynamic pages using React and JavaScript.}
      \resumeItem{Created the server side endpoints and databases using Flask, Python, and SQL.}
      \resumeItem{Implemented a search engine using tf-idf scoring and map-reduce techniques.}
      % \resumeItem{Deploy to Amazon Web Service using Nginx.}
    \resumeItemListEnd
    \vspace{-20pt}
    \resumeProjectHeading
    {\textbf{Probing into the Reason behind Wasserstein GAN’s Success} $|$ \emph{Machine Learning Course Project}}{Fall Semester, 2021}
    \resumeItemListStart
      \resumeItem{Implemented Wasserstein in Generative Adversarial Network, as well as WGAN with gradient penalty.}
      \resumeItem{Stated the problem of the original GAN loss function and analysis the modifications made by WGAN mathematically.}
      \resumeItem{Compared the stability of WGAN's loss function with the widely-used DCGAN using FID scores.}
    \resumeItemListEnd
        % \vspace{-15pt}
      % \resumeProjectHeading
      %   {\textbf{Arceus--Pokemon Gold Gym Environment} $|$ \emph{Personal Project}}{Winter Semester, 2022}
      %   \resumeItemListStart
      %     \resumeItem{Build the Gym environment for traning agents to play the \textit{Pokemon} game using keyboard inputs.}
      %     \resumeItem{Implement several RGB wrappers for data augmentation to train a robust agent.}
      %     \resumeItem{Scale-up the training environment by applying multi-processing, increase the total throughput by 5x.}
      %   \resumeItemListEnd

    \resumeSubHeadingListEnd
\vspace{-15pt}

%-----------INVOLVEMENT---------------
% \section{Honors \& Awards}
%     \vspace{-5pt}
%     \resumeSubHeadingListStart
%         \resumeProjectHeading{\textbf{Dean's List} $|$ University of Michigan}{Every Semester}
%     \resumeSubHeadingListEnd

%
%-----------PROGRAMMING SKILLS-----------
\section{Technical Skills}
 \begin{itemize}[leftmargin=0.15in, label={}]
    \small{\item{
     \textbf{Skills}{: Propmt Engineering, Natural Language Processing, Reinforcement Learning, Computer Vision} \\
     \textbf{Software and Libraries}{: Pytorch, Hugging Face, Gym, Unreal, Ray, Docker, React, Selenium, Unity} \\
     \textbf{Programming Language}{: Python, Java, JavaScript, SQL, C++} \\
    }}
 \end{itemize}
 \vspace{-16pt}




\end{document}
