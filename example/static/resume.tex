\documentclass[letterpaper,11pt]{article}

\usepackage{latexsym}
\usepackage[empty]{fullpage}
\usepackage{titlesec}
\usepackage{marvosym}
\usepackage[usenames,dvipsnames]{color}
\usepackage{verbatim}
\usepackage{enumitem}
\usepackage[colorlinks=true, linkcolor=blue]{hyperref}
\usepackage{fancyhdr}
\usepackage[english]{babel}
\usepackage{tabularx}
\usepackage{fontawesome5}
\usepackage{multicol}
\setlength{\multicolsep}{-3.0pt}
\setlength{\columnsep}{-1pt}
\input{glyphtounicode}


%----------FONT OPTIONS----------
% sans-serif
% \usepackage[sfdefault]{FiraSans}
% \usepackage[sfdefault]{roboto}
% \usepackage[sfdefault]{noto-sans}
% \usepackage[default]{sourcesanspro}

% serif
\usepackage{CormorantGaramond}
\usepackage{charter}


\pagestyle{fancy}
\fancyhf{} % clear all header and footer fields
\fancyfoot{}
\renewcommand{\headrulewidth}{0pt}
\renewcommand{\footrulewidth}{0pt}

% Adjust margins
\addtolength{\oddsidemargin}{-0.6in}
\addtolength{\evensidemargin}{-0.5in}
\addtolength{\textwidth}{1.19in}
\addtolength{\topmargin}{-.7in}
\addtolength{\textheight}{1.4in}

\urlstyle{same}

\raggedbottom
\raggedright
\setlength{\tabcolsep}{0in}

% Sections formatting
\titleformat{\section}{
  \vspace{-8pt}\raggedright\Large\bfseries
}{}{0em}{}[\color{black}\titlerule \vspace{-5pt}]

% Ensure that generate pdf is machine readable/ATS parsable
\pdfgentounicode=1

%-------------------------
% Custom commands
\newcommand{\resumeItem}[1]{
  \item\small{
    {\raggedright #1 \vspace{-2pt}}
  }
}

\newcommand{\classesList}[4]{
    \item\small{
        {#1 #2 #3 #4 \vspace{-2pt}}
  }
}

\newcommand{\resumeSubheading}[5]{
  \vspace{-2pt}\item
    \begin{tabular*}{1.0\textwidth}[t]{l@{\extracolsep{\fill}}r}
      \textbf{#1} \textit{#2} & \textbf{\small #3} \\
      \textit{\small#4} & \textit{\small #5} \\
    \end{tabular*}\vspace{-7pt}
}

\newcommand{\resumeSubSubheading}[2]{
    \item
    \begin{tabular*}{0.97\textwidth}{l@{\extracolsep{\fill}}r}
      \textit{\small#1} & \textit{\small #2} \\
    \end{tabular*}\vspace{-7pt}
}

\newcommand{\resumeProjectHeading}[2]{
    \item
    \begin{tabular*}{1.001\textwidth}{l@{\extracolsep{\fill}}r}
      \small#1 & \textbf{\small #2}\\
    \end{tabular*}\vspace{-7pt}
}

\newcommand{\resumeSubItem}[1]{\resumeItem{#1}\vspace{-4pt}}

\renewcommand\labelitemi{$\vcenter{\hbox{\tiny$\bullet$}}$}
\renewcommand\labelitemii{$\vcenter{\hbox{\tiny$\bullet$}}$}

\newcommand{\resumeSubHeadingListStart}{\begin{itemize}[leftmargin=0.0in, label={}]}
\newcommand{\resumeSubHeadingListEnd}{\end{itemize}}
\newcommand{\resumeItemListStart}{\begin{itemize}}
\newcommand{\resumeItemListEnd}{\end{itemize}\vspace{-3pt}}

%-------------------------------------------
%%%%%%  RESUME STARTS HERE  %%%%%%%%%%%%%%%%%%%%%%%%%%%%


\begin{document}

%----------HEADING----------
% \begin{tabular*}{\textwidth}{l@{\extracolsep{\fill}}r}
%   \textbf{\href{http://sourabhbajaj.com/}{\Large Sourabh Bajaj}} & Email : \href{mailto:sourabh@sourabhbajaj.com}{sourabh@sourabhbajaj.com}\\
%   \href{http://sourabhbajaj.com/}{http://www.sourabhbajaj.com} & Mobile : +1-123-456-7890 \\
% \end{tabular*}

\begin{center}
    {\Large \textbf{Yuqi Xie} } \\ \vspace{1pt}
    1859 Shirley Ln Apt B7, Ann Arbor, MI, 48105 \\ \vspace{1pt}
    \small \raisebox{-0.1\height}\faPhone\ 734-510-4285 ~ \href{mailto:xieleo@umich.edu}{\raisebox{-0.2\height}\faEnvelope\  \underline{xieleo@umich.edu}} ~ 
    %\href{https://linkedin.com/in//}{\raisebox{-0.2\height}\faLinkedin\ \underline{linkedin.com/in/username}}  ~
    \href{https://xieleo5.github.io}{\raisebox{-0.2\height}\faGithub\ \underline{xieleo5.github.io}}
    \vspace{-8pt}
\end{center}


%-----------EDUCATION-----------
\section{Education}
  \resumeSubHeadingListStart
    \resumeSubheading
      {University of Michigan, Ann Arbor}{}{Sep. 2021 -- Apr. 2023}
      {Bachelor of Computer Science - College of Engineering}{Michigan, U.S.}
    \begin{itemize}
      \item \small Highlighted Courses: Machine Learning, Computer Vision, Operating System, Web Systems, XR Development
      \item \small Research Interests: Reinforcement Learning, Embodied agents, Foundation Models, Multi-modal Learning, Robotics
      \item \small GPA: 3.96/4.0
    \end{itemize}
    \vspace*{5.0\multicolsep}
  \resumeSubHeadingListEnd
  \resumeSubHeadingListStart
    \resumeSubheading
      {Shanghai Jiao Tong University}{}{Sep. 2019 -- Aug. 2023}
      {Bachelor of Electrical and Computer Engineering - UM-SJTU Joint Institute}{Shanghai, China}
  \resumeSubHeadingListEnd

%-----------RESEARCH----------- 
\section{Research Experience}
  \resumeSubHeadingListStart
    \resumeSubheading
    {Collaboration with NVIDIA and UT Austin}{}{April 2022 - Present}
    {Undergraduate Research Assistant $|$ \small Advised by Prof. Yuke Zhu, Dr. Linxi "Jim" Fan}{Remote}
    \resumeItemListStart
      \resumeItem{
        \textbf{Development For \textsc{MineDojo2}} 
        
        \textsc{MineDojo} is a framework built on the popular \textit{Minecraft} game that features a simulation suite with thousands of diverse open-ended tasks and an Internet-scale knowledge base. 
        My contributions include: 
        \vspace*{-2pt}
        \begin{itemize}[leftmargin=*]
          \item Dramatically modify the infrastructure, make the interface more research friendly and the environment more scalable for training a large-scale open-ended agent.
          \item Implement a plugin to support a universal computer using interface instead of cheating commands for an agent to finish zero-shot tasks like a human.
          \item Increase the observation data transition speed between the game and the agent by 2x using on grpc protocol. 
          \item Enable GPU acceleration for the rendering process of \textit{Minecraft} game during training inside Docker containers.
        \end{itemize}
      }
      \resumeItem{
        \textbf{Embodied AI agent based on \textsc{MineDojo2}}

        We aim to build a general-purpose embodied agent that can finish a wider range of zero-shot tasks under various circumstances in the game based on \textsc{MineDojo2} framework. My contributions include:
        \vspace*{-2pt}
        \begin{itemize}[leftmargin=*]
          \item Create baseline task suites using the state-of-the-art OpenAI Video-PreTraining model.
          \item Apply Internet-scale pre-trained models on 300K Youtube videos in \textsc{MineDojo} to retrieve expert actions from human operations as data for large-scale training.
          \item Experiment with large-scale multi-modal Transformers on zero-shot human language driving \textit{Minecraft} tasks.
        \end{itemize}
      }
    \resumeItemListEnd  
    \resumeSubheading
    {SOCR MDP Team}{}{Winter and Fall Semester, 2022}
      {Undergraduate Research Assistant $|$ \small Advised by Prof. Ivo D. Dinov, Prof. Simeone Marino}{University of Michigan}
      \resumeItemListStart
        \resumeItem{\textbf{Datasifter---Sensitive Information Obfuscator}

        Datasifter is an obfuscator to prevent patients' sensitive information to be leaked from Hospital databases. The obfuscation is based on Machine Learning and Natural Language Processing technologies. My contributions includes:
        \vspace*{-2pt}
        \begin{itemize}[leftmargin=*]
          \item Experiment with pre-trained NLP models and ML metrics to implement the obfuscation part.
          \item Implement a script based on ML metrics to evaluate the effectiveness of data synthesis and data obfuscation.
          \item Improve the speed of the app by 10x by applying multi-thread programming.
        \end{itemize}
        }
    \resumeItemListEnd
    
  \resumeSubHeadingListEnd
\vspace{-15pt}
%
%-----------PROJECTS-----------
\section{Projects}
    \vspace{-5pt}
    \resumeSubHeadingListStart
      \resumeProjectHeading
        {\textbf{Probing into the Reason behind Wasserstein GAN’s Success} $|$ \emph{Machine Learning Course Project}}{Fall Semester, 2021}
        \resumeItemListStart
          \resumeItem{Implement Wasserstein in Generative Adversarial Network, as well as WGAN with gradient penalty.}
          \resumeItem{State the problem of the original GAN loss function and analysis the modifications made by WGAN mathematically.}
          \resumeItem{Train WGAN on three datasets: LSUN Bedroom, CelebA, and Animefaces. Compare the stability of WGAN's loss function with the widely-used DCGAN using FID scores.}
        \resumeItemListEnd
        \vspace{-15pt}
      \resumeProjectHeading
        {\textbf{Arceus--Pokemon Gold Gym Environment} $|$ \emph{Personal Project}}{Winter Semester, 2022}
        \resumeItemListStart
          \resumeItem{Build the Gym environment for traning agents to play the \textit{Pokemon} game using keyboard inputs.}
          \resumeItem{Implement several RGB wrappers for data augmentation to train a robust agent.}
          \resumeItem{Scale-up the training environment by applying multi-processing, increase the total throughput by 5x.}
        \resumeItemListEnd
    \resumeSubHeadingListEnd
\vspace{-15pt}

%-----------INVOLVEMENT---------------
% \section{Honors \& Awards}
%     \vspace{-5pt}
%     \resumeSubHeadingListStart
%         \resumeProjectHeading{\textbf{Dean's List} $|$ University of Michigan}{Every Semester}
%     \resumeSubHeadingListEnd

%
%-----------PROGRAMMING SKILLS-----------
\section{Technical Skills}
 \begin{itemize}[leftmargin=0.15in, label={}]
    \small{\item{
     \textbf{Skills}{: Reinforcement Learning, Training Infrastructure Development, Computer Vision, Natural Language Processing} \\
     \textbf{Software and Libraries}{: Gym, Ray, Tianshou, Docker, Pytorch, Gradle, Selenium, Mixin, GRPC, Unity, Unreal} \\
     \textbf{Programming Language}{: Python, Java, JavaScript, C++, C\#} \\
    }}
 \end{itemize}
 \vspace{-16pt}




\end{document}
